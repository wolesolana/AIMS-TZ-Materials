\documentclass[10pt,t,xcolor=dvipsnames]{beamer}

\usetheme{Antibes}
\usecolortheme{structure}
%\usepackage[utf8x]{inputenc}
%\usepackage{default}
%\usecolortheme{albatross}
%\usecolortheme{lily}
%\usecolortheme{sidebartab}
%\usecolortheme{crane}
%\usecolortheme{orchid}
%\usecolortheme{albatross}
%\usecolortheme{beetle}
%\usecolortheme{dove}
%\usecolortheme{fly}
%\usecolortheme{seagull}
%\usecolortheme{dolphin}
%\usecolortheme{rose}
\usepackage{graphics}
% \usepackage[pdftex]{graphicx}
\usepackage{graphicx}
\usepackage{xcolor}
\usepackage{color}
%\usepackage{url}
\usepackage{hyperref}
%\usepackage[obeyspaces]{url}
\usepackage{amssymb,amsmath} % for the bold symbol command
\usepackage{booktabs} % toprule etc in tables
\usepackage{mathrsfs}
\usepackage{listings}
\lstset{ %
  backgroundcolor=\color{white},
  basicstyle=\footnotesize,
  breakatwhitespace=false,
  breaklines=true,
  captionpos=b,
  commentstyle=\color{green},
  escapeinside={\%*}{*)},
  extendedchars=true,
  frame=single,
  keywordstyle=\color{blue},
  language=bash,
  numbers=left,
  numbersep=5pt,
  numberstyle=\tiny\color{gray},
  rulecolor=\color{black},
  showspaces=false,
  showstringspaces=false,
  showtabs=false,
  stepnumber=2,
  stringstyle=\color{red},
  tabsize=2,
  title=\lstname,
  morekeywords={not,\},\{,preconditions,effects },
  deletekeywords={time}
}
%\usepackage{minted}
%\usepackage{beamerthemebars}
%\usepackage{ragged2e}
%\usepackage{lipsum}

\setbeamercolor{structure}{fg=RawSienna!50!black}
\renewcommand{\raggedright}{\leftskip=0pt \rightskip=0pt plus 0cm}
\logo{\includegraphics[scale=0.05]{../../../TWAssets/TW_Colour_Logos_trans_green.png}}

\title{ An Introduction to Django }
\titlegraphic{\includegraphics[scale=0.15]{../../../TWAssets/TW_Colour_Logos_trans_green.png}}

\author{ Patrick Turley \& 'Wole Solana }
\begin{document}
\nocite*{}
\frame [c, plain]{\titlepage}
%-------------------------------Slide1-------------------------------
\section{Introduction}
\begin{frame}
\frametitle{What is Django?}
\pause
\begin{columns}[l]
\column{0.5\textwidth}
\begin{itemize}[<+->]
\item \alert{Django} is an open source web application (web app) framework written in Python, and follows the \alert{Model-View-Controller} (MVC) design pattern.
\item A \alert{web framework} can be loosely defined as a set of components that assist in building web apps faster and easier.
\item Other frameworks include Rails (Ruby), Drupal (PHP) and Spring (Java)
\item Popular sites built with Django include NASA, Instagram, The Guardian.
\end{itemize}
\column{0.5\textwidth}
\begin{figure}
\centering
\includegraphics[scale=0.3]{../images/django.jpg} 
\end{figure}
\end{columns}
\end{frame}
%------------------------------Slide2---------------------------------
\begin{frame}[fragile]
\frametitle{Why use a web framework?}
\pause
\begin{columns}[l]
\column{0.5\textwidth}
\vspace{-1cm}
\begin{figure}
\centering
\includegraphics[scale=0.125]{../images/webapp_building_blocks.png}
\caption{\footnotesize{Schematic diagram of a web app}}
\end{figure}
\column{0.5\textwidth}
\begin{itemize}[<+->]
\item It enables us to build web apps faster by focusing on the unique functionality of our web app rather than the infrastucture
\item Most provide us with with libraries and templates which we can reuse, thus enhancing our productivity.
\end{itemize}
\end{columns}
\end{frame}
%------------------------------Slide3----------------------------------
\section{The MVC design paradigm}
\begin{frame}[fragile]
\frametitle{Model-View-Controller}
\pause
\begin{columns}[l]
\column{0.4\textwidth}
\begin{itemize}[<+->]
\item A design pattern (which extends OOP) that helps us better organise our web app by \alert{separating concerns}
\item The \textbf{model} contains \alert{data and logic}, the \textbf{view} forms the \alert{user interface} (what the user sees and the \textbf{controller} manages the \alert{interaction} between the model and view.
\end{itemize}
\column{0.5\textwidth}
\vspace{-0.5cm}
\begin{figure}
\hspace*{-0.5cm}
\centering
\includegraphics[scale=0.4]{../images/mvc-02.jpg} 
\caption{The MVC pattern}
\end{figure}
\end{columns}
\end{frame}
%--------------------------------------------Slide7------------------------------------------------------------------
\section{Further Reading}
\begin{frame}
\frametitle{Further Reading}
% \bibliographystyle{plain}
% \bibliography{DevelopmentPractices.bib}
\end{frame}
\end{document}

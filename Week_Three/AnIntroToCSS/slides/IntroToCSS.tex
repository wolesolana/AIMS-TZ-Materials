\documentclass[10pt,t,xcolor=dvipsnames]{beamer}

\usetheme{Antibes}
\usecolortheme{structure}
%\usepackage[utf8x]{inputenc}
%\usepackage{default}
%\usecolortheme{albatross}
%\usecolortheme{lily}
%\usecolortheme{sidebartab}
%\usecolortheme{crane}
%\usecolortheme{orchid}
%\usecolortheme{albatross}
%\usecolortheme{beetle}
%\usecolortheme{dove}
%\usecolortheme{fly}
%\usecolortheme{seagull}
%\usecolortheme{dolphin}
%\usecolortheme{rose}
\usepackage{graphics}
% \usepackage[pdftex]{graphicx}
\usepackage{graphicx}
\usepackage{xcolor}
\usepackage{color}
%\usepackage{url}
\usepackage{hyperref}
%\usepackage[obeyspaces]{url}
\usepackage{amssymb,amsmath} % for the bold symbol command
\usepackage{booktabs} % toprule etc in tables
\usepackage{mathrsfs}
\usepackage{listings}
\lstset{ %
  backgroundcolor=\color{white},
  basicstyle=\footnotesize,
  breakatwhitespace=false,
  breaklines=true,
  captionpos=b,
  commentstyle=\color{green},
  escapeinside={\%*}{*)},
  extendedchars=true,
  frame=single,
  keywordstyle=\color{blue},
  language=bash,
  numbers=left,
  numbersep=5pt,
  numberstyle=\tiny\color{gray},
  rulecolor=\color{black},
  showspaces=false,
  showstringspaces=false,
  showtabs=false,
  stepnumber=2,
  stringstyle=\color{red},
  tabsize=2,
  title=\lstname,
  morekeywords={not,\},\{,preconditions,effects },
  deletekeywords={time}
}
%\usepackage{minted}
%\usepackage{beamerthemebars}
%\usepackage{ragged2e}
%\usepackage{lipsum}

\setbeamercolor{structure}{fg=Plum!50!black}
\renewcommand{\raggedright}{\leftskip=0pt \rightskip=0pt plus 0cm}
\logo{\includegraphics[scale=0.05]{../../../TWAssets/TW_Colour_Logos_trans_black.png}
      \includegraphics[scale=0.05]{../../../TWAssets/aimsgh-vertical2.png}
      }

\title{ An Introduction to CSS }
\titlegraphic{\includegraphics[scale=0.15]{../../../TWAssets/TW_Colour_Logos_trans_black.png}
              \includegraphics[scale=0.15]{../../../TWAssets/aimsgh-vertical2.png}
              }

\author{ Faris Mohammed \& 'Wole Solana }
\begin{document}
\nocite*{}
\frame [c, plain]{\titlepage}
%-------------------------------Slide1-------------------------------
\section{Making things pretty}
\begin{frame}
\frametitle{What is CSS?}
\pause
\begin{itemize}[<+->]
\item \alert{C}ascading \alert{S}tyle \alert{S}heets is a programming language used to modify the how the design of a web page written in HTML looks and is formatted. It is the \textit{``prettifier''} of our webpages.
\item You can write your own CSS file, describing how you want the different HT§ML elements on your page to look like or...
\item You can use some CSS frameworks which are readily available online and add it to your project.
\item You can then add some custom design and formatting as required.
\item \alert{Beware} with ``overcustomization'' of frameworks imported into your project. Customization should be done sparingly.
\end{itemize}
\end{frame}
%------------------------------Slide2----------------------------------
\section{Available CSS tools}
\begin{frame}[fragile]
\frametitle{Some CSS frameworks}
\pause
There are quite a number of CSS frameworks online that you can add to your project. Some of these include:
\begin{itemize}[<+->]
\item Bootstrap CSS: \texttt{\href{bootstrap_link}{http://getbootstrap.com/}}
\item Zurb: \texttt{\href{zurb_link}{http://foundation.zurb.com/}}
\item Kube: \texttt{\href{kube_link}{http://imperavi.com/kube/}}
\item Pure CSS: \texttt{\href{pure_link}{http://purecss.io/}}
\item You can search online for other frameworks and how to add them to your app
\end{itemize}
\end{frame}
%------------------------------Slide3---------------------------------
\begin{frame}[fragile]
\frametitle{Adding CSS to your project}
\pause
\begin{columns}[l]
\column{0.5\textwidth}
\begin{itemize}[<+->]
\item Create a \texttt{``static''} folder.
\item In it, place the folders from whatever CSS framework you have decided to use so your file structure looks something like the diagram on the right.
\item Import the file path(s) into your HTML \texttt{header} and \texttt{script} blocks and add the line to your html views:
\begin{lstlisting}
.
\end{lstlisting}
\end{itemize}
\pause
\column{0.5\textwidth}
\begin{lstlisting}
-- static
│  -- css
│  -- img
│  -- js
\end{lstlisting}
\end{columns}
\end{frame}
%--------------------------------------------Slide7------------------------------------------------------------------
\section{A word on Javascript}
\begin{frame}[fragile]
\frametitle{Why Javascript?}
\begin{itemize}[<+->]
\item \textit{\alert{Javascript}} (not \textit{\alert{Java}}) is a dynamic programming language, and is commonly used in web browsers for client side scripting of user interactions.
\item There are other implementations of javascript that are widely used in creating web apps including
\begin{itemize}
\item Angular (\texttt{\href{angular_link}{https://angularjs.org/}}),
\item Ember (\texttt{\href{ember_link}{http://emberjs.com/}}),
\item Knockout (\texttt{\href{knockout_link}{http://knockoutjs.com/}}), and
\item Backbone (\texttt{\href{backbone_link}{http://backbonejs.org/}}) JS.
\end{itemize}
\item Javascript is also used for some server side applications. NodeJS (\texttt{\href{node_link}{https://nodejs.org/}}) is widely used in that regard.
\end{itemize}
\end{frame}
%--------------------------------------------Slide13------------------------------------------------------------------
\section{Online resources}
\begin{frame}
\frametitle{Further Reading}
% \bibliographystyle{plain}
% \bibliography{DevelopmentPractices.bib}
\end{frame}
\end{document}

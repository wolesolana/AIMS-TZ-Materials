\documentclass[10pt,t,sans,mathsans,xcolor=dvipsnames]{beamer}

\usetheme{Antibes}
\usecolortheme{structure}
%\usepackage[utf8x]{inputenc}
%\usepackage{default}
%\usecolortheme{albatross}
%\usecolortheme{lily}
%\usecolortheme{sidebartab}
%\usecolortheme{crane}
%\usecolortheme{orchid}
%\usecolortheme{albatross}
%\usecolortheme{beetle}
%\usecolortheme{dove}
%\usecolortheme{fly}
%\usecolortheme{seagull}
%\usecolortheme{dolphin}
%\usecolortheme{rose}
\usepackage{graphics}
%\usepackage[pdftex]{graphicx}
%\usepackage{graphicx}
\usepackage{xcolor}
\usepackage{color}
%\usepackage{url}
\usepackage{hyperref}
%\usepackage[obeyspaces]{url}
\usepackage{amssymb,amsmath} % for the bold symbol command
\usepackage{booktabs} % toprule etc in tables
\usepackage{mathrsfs}
\usepackage{listings}
\lstset{ %
  backgroundcolor=\color{white},
  basicstyle=\footnotesize,
  breakatwhitespace=false,
  breaklines=true,
  captionpos=b,
  commentstyle=\color{green},
  escapeinside={\%*}{*)},
  extendedchars=true,
  frame=single,
  keywordstyle=\color{blue},
  language=bash,
  numbers=left,
  numbersep=5pt,
  numberstyle=\tiny\color{gray},
  rulecolor=\color{black},
  showspaces=false,
  showstringspaces=false,
  showtabs=false,
  stepnumber=2,
  stringstyle=\color{red},
  tabsize=2,
  title=\lstname,
  morekeywords={not,\},\{,preconditions,effects },
  deletekeywords={time}
}
%\usepackage{minted}
%\usepackage{beamerthemebars}
%\usepackage{ragged2e}
%\usepackage{lipsum}

\setbeamercolor{structure}{fg=OliveGreen!50!black}
\renewcommand{\raggedright}{\leftskip=0pt \rightskip=0pt plus 0cm}

\title{ Python Coding Dojo }

\author{ Brain Leke Betetchouh \& 'Wole Solana }
\begin{document}
\frame [c, plain]{\titlepage}
%-------------------------------Slide1----------------------------------------------------------------------------------------
\section{Introduction}
\begin{frame}
\frametitle{What are Katas?}
\end{frame}
%------------------------------Slide2-------------------------------------------------------------------------------------------
\section{The Katas}
\begin{frame}[fragile]
\frametitle{Exercise One: Fizz Buzz}
\begin{itemize}
\item Stage One:
\begin{itemize}
\item Write a program that prints the numbers from 1 to 100. But for multiples of three print ``\texttt{Fizz}" instead of the number and for the multiples of five print ``\texttt{Buzz}". For numbers which are multiples of both three and five print ``\texttt{FizzBuzz}"
\end{itemize}
\item Stage Two:
\begin{itemize}
\item A number is fizz if it is divisible by 3 or if it has a 3 in it
\item A number is buzz if it is divisible by 5 or if it has a 5 in it
\end{itemize}
\end{itemize}
\end{frame}
%------------------------------Slide3-----------------------------------
\begin{frame}[fragile]
\frametitle{Exercise Two: Prime Factors}
\begin{itemize}
\item Stage One:
\begin{itemize}
\item Factorize a positive integer number into its prime factors e.g.
\begin{verbatim}
2 	: [2]
4 	: [2, 2]
9 	: [3, 3]
20 : [2, 2, 5]
\end{verbatim}
\end{itemize}
\item Stage Two:
\begin{itemize}
\item Return the prime factors without repetition e.g.
\begin{verbatim}
20 : [2, 5]
32 : [2]
40 : [2, 5]
\end{verbatim}
\end{itemize}
\end{itemize}
\end{frame}
%---------------------------------------------Slide4------------------------------------------------------------
\begin{frame}[fragile]
\frametitle{Exercise Three: Roman Numerals}
\begin{itemize}
\item Stage One:
\begin{itemize}
\item The Romans wrote numbers using letters - I, V, X, L, C, D, M. (notice these letters have lots of straight lines and are hence easy to hack into stone tablets). The Kata says you should write a function to convert from normal numbers to Roman Numerals: e.g.
\begin{verbatim}
1 	--> I
10 	--> X
7 	--> VII
\end{verbatim}
etc. There is no need to be able to convert numbers larger than about 3000.
\end{itemize}
\item Stage Two:
\begin{itemize}
\item Write a function to convert in the other direction, i.e. roman numeral to digits
\end{itemize}
\end{itemize}
\end{frame}
%---------------------------------------------Slide5----------------------------------------------------------------
\section{Guidelines}
\begin{frame}[fragile]%[allowframebreaks]
\frametitle{Remember...}
\begin{itemize}
\item Test first
\item Think in an Object Oriented Way
\end{itemize}
\end{frame}
\end{document}
